%Copyright 2014 Jean-Philippe Eisenbarth
%This program is free software: you can 
%redistribute it and/or modify it under the terms of the GNU General Public 
%License as published by the Free Software Foundation, either version 3 of the 
%License, or (at your option) any later version.
%This program is distributed in the hope that it will be useful,but WITHOUT ANY 
%WARRANTY; without even the implied warranty of MERCHANTABILITY or FITNESS FOR A 
%PARTICULAR PURPOSE. See the GNU General Public License for more details.
%You should have received a copy of the GNU General Public License along with 
%this program.  If not, see <http://www.gnu.org/licenses/>.

%Based on the code of Yiannis Lazarides
%http://tex.stackexchange.com/questions/42602/software-requirements-specification-with-latex
%http://tex.stackexchange.com/users/963/yiannis-lazarides
%Also based on the template of Karl E. Wiegers
%http://www.se.rit.edu/~emad/teaching/slides/srs_template_sep14.pdf
%http://karlwiegers.com


\documentclass{scrreprt}

\usepackage{pdfpages}
\usepackage{listings}
\usepackage[T1]{fontenc}
\usepackage{placeins}
\usepackage{float}
\usepackage{underscore}
\usepackage[bookmarks=true]{hyperref}
\usepackage[utf8]{inputenc}
\usepackage{graphicx}
\usepackage{subfigure}
\usepackage[english]{babel}
\hypersetup{
	bookmarks=false,    % show bookmarks bar?
	pdftitle={Software Requirement Specification},    % title
	pdfauthor={Adam-Ryan},                     % author
	pdfsubject={TeX and LaTeX},                        % subject of the document
	pdfkeywords={TeX, LaTeX, graphics, images}, % list of keywords
	colorlinks=true,       % false: boxed links; true: colored links
	linkcolor=blue,       % color of internal links
	citecolor=black,       % color of links to bibliography
	filecolor=black,        % color of file links
	urlcolor=blue,        % color of external links
	linktoc=page            % only page is linked
}%
\def\myversion{1 }
\date{}
%\title
\usepackage{hyperref}
\begin{document}
	
	\begin{flushright}
		\rule{16cm}{5pt}\vskip1cm
		\begin{bfseries}
			\Huge{Tutorial 1\\}
			\vspace{1.9cm}
			for\\
			\vspace{1.9cm}
			Data Mining - Exploration and Pre-processing
			\vspace{1.9cm}
			\LARGE{Version \myversion}\\
			\vspace{1.9cm}
			Adam Ryan (14395076)\\
			\vspace{1.9cm}
			COMP47530\\
			\vspace{1.9cm}
			\today\\
		\end{bfseries}
	\end{flushright}
	
	\tableofcontents
	
	\chapter{Questions and Answers}\label{Intro}
	
	
\section{Exercise 1 - Questions}\label{E1Q}
An analyst collects surveys from different participants about their likes and dislikes.
Subsequently, the analyst corrects erroneous or missing entries, uploads the data to a data warehouse, and designs a recommendation algorithm on this basis.
\\
\\
Which of the following actions represent data collection, data preprocessing, and data analysis?
\begin{itemize}
\item Conducting surveys and uploading to a database.
\item Correcting missing entries.
\item Designing a recommendation algorithm.
\end{itemize}

	
\section{Question 1 - Answer}







\newpage
\section{Exercise 2 - Questions}\label{E2Q}

One of the important aspects of data collections is that they contain a wide variety of data types, which should be taken into account during the analysis. From the analysis point of view, we can distinguish two broad categories of data: Non-dependency-oriented data, and Dependency-oriented data.
\\
\\
\textbf{Non-dependency-oriented data}: The data records do not have any specific dependencies between either the data items or the attributes. A record is referred to as data point, instance, example, transaction, entity, tuple, object, or feature- vector. Each record contains a set of fields, which are also referred to as attributes, dimensions, and features. Non-dependency-oriented data is the simplest form of data and typically refers to multidimensional data.
We can have attributes of different types of data. These include numeric data, categorical data, binary data, text data.
\\
\\
\textbf{Dependency-oriented data}: the data contains implicit or explicit relationships. In the case of implicit dependencies, the relationship is not expressed in the data. For example consecutive reading of temperature values, that are close in time, are more likely to be similar. Explicit dependencies refer to graph or network data in which edges are used to specify explicit relationships.
\\
\\
Different types of dependency-oriented data are: Times-Series Data, Discrete Sequences and Strings, Spatial Data, Spatio-temporal Data, and Network and Graph Data.
\\
\\
The same analyst obtains medical notes from a physician for data mining purposes, and then transforms them into a table containing the medicines prescribed for each patient. What is the data type of
\begin{enumerate}
\item The original data,
\item The transformed data?
\item What is the process of transforming the data to the new format called?
\end{enumerate}


	

\section{Exercise 2 - Answers}















	
\newpage	
\section{Exercise 3 - Questions}
Suppose that the data for analysis includes the attribute age. The age values for the data tuples are (in increasing order):
\\
\\
13, 15, 16, 16, 19, 20, 20, 21, 22, 22, 25, 25, 25, 25, 30, 33, 33, 35, 35, 35, 35, 36, 40, 45, 46, 52, 70
\begin{enumerate}
\item What is the mean of the data? What is the median?
\item What is the mode of the data? Comment on the data’s modality (i.e., bimodal, trimodal, etc.). 
\item What is the midrange of the data?
\item Can you find (roughly) the first quartile (Q1) and the third quartile (Q3) of the data?
\item Give the five-number summary of the data.
\item Show a boxplot of the data.
\item How is a quantile-quantile plot different from a quantile plot?
\end{enumerate}




\section{Exercise 3 - Answers}	












	
\newpage
\section{Exercise 4 - Question}
The file AutoMpg question1.csv contains data related to cars, such as horsepower, weight, car name, and so on. Unfortunately, some of the values for the horsepower and origin columns were not properly recorded. Can you tell how many missing values are there for each one of these columns? Write the answer in your report.

\begin{enumerate}
\item Replace the missing horsepower values with the average of this column.
\item Replace the missing origin values with the minimum of this column
\item Save the generated data file to ./output/question1 out.csv
\end{enumerate}

When saving the generated data, pay extra attention to the columns included in the file (hint: if you are using pandas, take a look at the arguments of the to\_csv function).


\section{Exercise 4 - Answers}
	
	
	
	
	
	
	
	
\newpage
\section{Exercise 5 - Questions}

The files AutoMpg question2 a.csv and AutoMpg question2\_b.csv contain similar pieces of information about car models. There are some differences between the 2 files. What you need to do is:
\begin{enumerate}
\item The dataset A has an attribute called car name, whereas the dataset B has an attribute called name. Rename the name attribute to car name (unintended tongue twister!).
\item The dataset B has an attribute called other, which is not present in the dataset A. Create an attribute called other in the dataset A and assign it a default value of 1.
\item Concatenate dataset A and B together, and just like in question 1, save the resulting file to ./output/question2\_out.csv.
\end{enumerate}





\section{Exercise 5 - Answers}

	
\end{document}
